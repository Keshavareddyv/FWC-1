\documentclass[12pt]{article}

\usepackage[a4paper,margin=1in]{geometry}
\usepackage{amsmath,amssymb}
\usepackage{graphicx}

\setlength{\parindent}{0pt}
\setlength{\parskip}{0pt}

\begin{document}

%================ HEADER =================%
\begin{center}
\includegraphics[width=4.5cm]{iiitb_logo.png}\\[0.3cm]
\textbf{\Large Keshava Reddy V}\\
\textbf{ID: COMETFWC054}
\end{center}

\vspace{0.6cm}

\textbf{46th IMO 2005}

\vspace{0.6cm}

%================ PROBLEMS =================%

\textbf{Problem 1.}
Six points are chosen on the sides of an equilateral triangle $ABC$:
$A_1, A_2$ on $BC$, $B_1, B_2$ on $CA$ and $C_1, C_2$ on $AB$, such that they are
the vertices of a convex hexagon $A_1A_2B_1B_2C_1C_2$ with equal side lengths.
Prove that the lines $A_1B_2$, $B_1C_2$ and $C_1A_2$ are concurrent.

\vspace{0.35cm}

\textbf{Problem 2.}
Let $a_1, a_2, \ldots$ be a sequence of integers with infinitely many positive
and negative terms.
Suppose that for every positive integer $n$ the numbers
$a_1, a_2, \ldots, a_n$ leave $n$ different remainders upon division by $n$.
Prove that every integer occurs exactly once in the sequence
$a_1, a_2, \ldots$.

\vspace{0.35cm}

\textbf{Problem 3.}
Let $x, y, z$ be three positive real numbers such that $xyz \ge 1$.
Prove that
\[
\frac{x^5 - x^2}{x^5 + y^2 + z^2}
+
\frac{y^5 - y^2}{x^2 + y^5 + z^2}
+
\frac{z^5 - z^2}{x^2 + y^2 + z^5}
\ge 0.
\]

\vspace{0.35cm}

\textbf{Problem 4.}
Determine all positive integers relatively prime to all the terms of the
infinite sequence
\[
a_n = 2^n + 3^n + 6^n - 1, \quad n \ge 1.
\]

\vspace{0.35cm}

\textbf{Problem 5.}
Let $ABCD$ be a fixed convex quadrilateral with $BC = DA$ and $BC$ not parallel
to $DA$.
Let two variable points $E$ and $F$ lie on the sides $BC$ and $DA$,
respectively, and satisfy $BE = DF$.
The lines $AC$ and $BD$ meet at $P$, the lines $BD$ and $EF$ meet at $Q$,
the lines $EF$ and $AC$ meet at $R$.
Prove that the circumcircles of the triangles $PQR$, as $E$ and $F$ vary,
have a common point other than $P$.

\vspace{0.35cm}

\textbf{Problem 6.}
In a mathematical competition, in which 6 problems were posed to the
participants, every two of these problems were solved by more than
$\frac{2}{5}$ of the contestants.
Moreover, no contestant solved all the 6 problems.
Show that there are at least 2 contestants who solved exactly 5 problems
each.

\vfill
\begin{center}

\end{center}

\end{document}
