\documentclass[12pt]{article}

\usepackage{graphicx}
\usepackage{amsmath}
\usepackage{amssymb}

\setlength{\parindent}{0pt}

\begin{document}

\begin{center}
\includegraphics[width=4cm]{iiitb_logo.png}
\end{center}

\begin{center}
Keshava Reddy V\\
ID: COMETFWC054\\
46th International Mathematical Olympiad 2005
\end{center}

\textbf{Problem 1.}

Six points are chosen on the sides of an equilateral triangle $ABC$.
Points $A_1$ and $A_2$ lie on $BC$, points $B_1$ and $B_2$ lie on $CA$,
and points $C_1$ and $C_2$ lie on $AB$,
such that they are the vertices of a convex hexagon
$A_1A_2B_1B_2C_1C_2$ with equal side lengths.
Prove that the lines $A_1B_2$, $B_1C_2$, and $C_1A_2$ are concurrent.

\textbf{Problem 2.}

Let $a_1, a_2$ and subsequent terms be a sequence of integers
with infinitely many positive terms and infinitely many negative terms.
Suppose that for every positive integer $n$,
the numbers $a_1, a_2$ up to $a_n$ leave $n$ different remainders
upon division by $n$.
Prove that every integer occurs exactly once in the sequence.

\textbf{Problem 3.}

Let $x, y, z$ be positive real numbers such that $xyz >= 1$.
Prove that
$$
(x^5 - x^2)/(x^5 + y^2 + z^2)
+
(y^5 - y^2)/(x^2 + y^5 + z^2)
+
(z^5 - z^2)/(x^2 + y^2 + z^5)
>= 0.
$$

\textbf{Problem 4.}

Determine all positive integers that are relatively prime to all terms
of the infinite sequence defined by
$$
a_n = 2^n + 3^n + 6^n - 1,
$$
for all positive integers $n$.

\textbf{Problem 5.}

Let $ABCD$ be a fixed convex quadrilateral with $BC = DA$
and $BC$ not parallel to $DA$.
Let two variable points $E$ and $F$ lie on the sides $BC$ and $DA$
respectively such that $BE = DF$.
The lines $AC$ and $BD$ meet at $P$,
the lines $BD$ and $EF$ meet at $Q$,
and the lines $EF$ and $AC$ meet at $R$.
Prove that the circumcircles of triangles $PQR$,
as $E$ and $F$ vary,
have a common point different from $P$.

\textbf{Problem 6.}

In a mathematical competition with six problems,
every two of the problems were solved by more than $5/2$
of the contestants.
Moreover, no contestant solved all six problems.
Show that there are at least two contestants
who solved exactly five problems.

\end{document}
