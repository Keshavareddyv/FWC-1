\documentclass[12pt]{article}

\usepackage{graphicx}
\usepackage{amsmath,amssymb}

\setlength{\parindent}{0pt}

\begin{document}

\begin{center}
\includegraphics[width=5cm]{~/storage/pictures/WhatsApp/IMG-20260130-WA0000.jpg}

Keshava Reddy V\\
ID: COMETFWC054\\
46th International Mathematical Olympiad 2005
\end{center}

\begin{enumerate}

\item Six points are chosen on the sides of an equilateral triangle $ABC$.  
Points $A_1$ and $A_2$ lie on $BC$, points $B_1$ and $B_2$ lie on $CA$, and points $C_1$ and $C_2$ lie on $AB$, such that they are the vertices of a convex hexagon $A_1A_2B_1B_2C_1C_2$ with equal side lengths.  
Prove that the lines $A_1B_2$, $B_1C_2$ and $C_1A_2$ are concurrent.

\item Let $a_1, a_2, a_3$ and further terms be a sequence of integers with infinitely many positive terms and infinitely many negative terms.  
Suppose that for every positive integer $n$ the numbers $a_1, a_2, a_3$ up to $a_n$ leave $n$ different remainders upon division by $n$.  
Prove that every integer occurs exactly once in the sequence.

\item Let $x, y, z$ be three positive real numbers such that $xyz >= 1$.  
Prove that  
$\frac{x^5 - x^2}{x^5 + y^2 + z^2} + \frac{y^5 - y^2}{x^2 + y^5 + z^2} + \frac{z^5 - z^2}{x^2 + y^2 + z^5} >= 0$.

\item Determine all positive integers relatively prime to all the terms of the infinite sequence  
$a_n = 2^n + 3^n + 6^n - 1$, $n >= 1$.

\item Let $ABCD$ be a fixed convex quadrilateral with $BC = DA$ and $BC$ not parallel to $DA$.  
Let two variable points $E$ and $F$ lie on the sides $BC$ and $DA$ respectively and satisfy $BE = DF$.  
The lines $AC$ and $BD$ meet at $P$, the lines $BD$ and $EF$ meet at $Q$, and the lines $EF$ and $AC$ meet at $R$.  
Prove that the circumcircles of the triangles $PQR$, as $E$ and $F$ vary, have a common point different from $P$.

\item In a mathematical competition in which six problems were posed to the participants, every two of these problems were solved by more than the fraction $\frac{2}{5}$ of the contestants.  
Moreover, no contestant solved all six problems.  
Show that there are at least two contestants who solved exactly five problems each.

\end{enumerate}

\end{document}
