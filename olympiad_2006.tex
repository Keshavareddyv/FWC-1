\documentclass[12pt]{article}

\usepackage[a4paper,margin=1in]{geometry}
\usepackage{amsmath,amssymb}
\usepackage{graphicx}

\setlength{\parindent}{0pt}
\setlength{\parskip}{10pt}

\begin{document}

\begin{center}
\includegraphics[width=4.5cm]{iiitb_logo.png}

Keshava Reddy V  
ID: COMETFWC054  

47th International Mathematical Olympiad  
Slovenia 2006  

Language: English
\end{center}

12 July 2006

\textbf{Problem 1.}  
Let $ABC$ be a triangle with incenter $I$. A point $P$ in the interior of the triangle satisfies  
$PBA + PCA = PBC + PCB$.  

Show that $AP >= AI$, and that equality holds if and only if $P = I$.


\textbf{Problem 2.}  
Let $P$ be a regular $2006$-gon. A diagonal of $P$ is called good if its endpoints divide the boundary of $P$ into two parts, each composed of an odd number of sides of $P$. The sides of $P$ are also called good.  

Suppose $P$ has been dissected into triangles by $2003$ diagonals, no two of which have a common point in the interior of $P$. Find the maximum number of isosceles triangles having two good sides that could appear in such a configuration.


\textbf{Problem 3.}  
Determine the least real number $M$ such that  

$|ab(a^2 - b^2) + bc(b^2 - c^2) + ca(c^2 - a^2)| <= M(a^2 + b^2 + c^2)^2$  

holds for all real numbers $a$, $b$ and $c$.


Time allowed: 4 hours 30 minutes  
Each problem is worth 7 points

\newpage

13 July 2006

\textbf{Problem 4.}  
Determine all pairs $(x, y)$ of integers such that  

$1 + 2^x + 2^{2x+1} = y^2$.


\textbf{Problem 5.}  
Let $P(x)$ be a polynomial of degree $n > 1$ with integer coefficients and let $k$ be a positive integer.  

Define $Q(x) = P(P(P(x)))$ where $P$ is applied $k$ times.  

Prove that there are at most $n$ integers $t$ such that $Q(t) = t$.


\textbf{Problem 6.}  
Assign to each side $b$ of a convex polygon $P$ the maximum area of a triangle that has $b$ as a side and is contained in $P$.  

Show that the sum of the areas assigned to the sides of $P$ is at least twice the area of $P$.


Time allowed: 4 hours 30 minutes  
Each problem is worth 7 points

\end{document}
