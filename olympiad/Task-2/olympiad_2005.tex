\documentclass[12pt]{article}

\usepackage[draft=false]{graphicx}
\usepackage{amsmath}
\usepackage{amssymb}

\setlength{\parindent}{0pt}

\begin{document}

\begin{center}
\includegraphics[width=4cm]{iiitb_logo.png}
\end{center}

\begin{center}
Keshava Reddy V\\
ID: COMETFWC054\\
46th IMO 2005
\end{center}

\textbf{Problem 1.}
Six points are chosen on the sides of an equilateral triangle $ABC$:
$A_1, A_2$ on $BC$, $B_1, B_2$ on $CA$ and $C_1, C_2$ on $AB$,
such that they are the vertices of a convex hexagon
$A_1A_2B_1B_2C_1C_2$ with equal side lengths.
Prove that the lines $A_1B_2$, $B_1C_2$ and $C_1A_2$ are concurrent.

\textbf{Problem 2.}
Let $a_1, a_2, ...$ be a sequence of integers with infinitely many positive
and negative terms.
Suppose that for every positive integer $n$ the numbers
$a_1, a_2, ..., a_n$ leave $n$ different remainders upon division by $n$.
Prove that every integer occurs exactly once in the sequence
$a_1, a_2, ...$.

\textbf{Problem 3.}
Let $x, y, z$ be three positive real numbers such that $xyz >= 1$.
Prove that
\[
(x^5 - x^2)/(x^5 + y^2 + z^2)
+ (y^5 - y^2)/(x^2 + y^5 + z^2)
+ (z^5 - z^2)/(x^2 + y^2 + z^5) >= 0
\]

\textbf{Problem 4.}
Determine all positive integers relatively prime to all the terms of the
infinite sequence
\[
a_n = 2^n + 3^n + 6^n - 1, \; n >= 1
\]

\textbf{Problem 5.}
Let $ABCD$ be a fixed convex quadrilateral with $BC = DA$ and $BC$ not
parallel to $DA$.
Let two variable points $E$ and $F$ lie on the sides $BC$ and $DA$,
respectively, and satisfy $BE = DF$.
The lines $AC$ and $BD$ meet at $P$, the lines $BD$ and $EF$ meet at $Q$,
the lines $EF$ and $AC$ meet at $R$.
Prove that the circumcircles of the triangles $PQR$, as $E$ and $F$ vary,
have a common point other than $P$.

\textbf{Problem 6.}
In a mathematical competition, in which 6 problems were posed to the
participants, every two of these problems were solved by more than
$2/5$ of the contestants.
Moreover, no contestant solved all the 6 problems.
Show that there are at least 2 contestants who solved exactly 5 problems
each.

\begin{center}

\end{center}

\end{document}
