\documentclass[12pt]{article}
\usepackage{amsmath}
\usepackage{geometry}
\usepackage{graphicx}
\geometry{a4paper, margin=1in}

\begin{document}

\begin{center}
\includegraphics[width=6cm]{/data/data/com.termux/files/home/storage/dcim/MyAlbums/documents/IMG_20260202_154637.jpg}
\end{center}

\vspace{0.5cm}

\begin{center}
\textbf{Name: Keshava Reddy V} \\
\textbf{ID: COMETFWC054}
\end{center}

\vspace{0.5cm}

\begin{center}
{\Large \textbf{GATE CS 2010, Majority Function Implementation}}
\end{center}

\vspace{1cm}

\section*{Question}

For the Boolean function

$f(P,Q,R) = PQ + QR + PR$

Find the minterm expansion and implement in hardware.

\vspace{0.5cm}

\section*{Question Analysis}

• Output is 1 when at least two inputs are 1.  
• This is a 3-input Majority Function.  
• Canonical expansion required.

\vspace{0.5cm}

\section*{The Truth Table}

\[
\begin{array}{c c c c}
P & Q & R & F \\
0 & 0 & 0 & 0 \\
0 & 0 & 1 & 0 \\
0 & 1 & 0 & 0 \\
0 & 1 & 1 & 1 \\
1 & 0 & 0 & 0 \\
1 & 0 & 1 & 1 \\
1 & 1 & 0 & 1 \\
1 & 1 & 1 & 1 \\
\end{array}
\]

Minterms where $F = 1$:

$m_3, m_5, m_6, m_7$

\vspace{0.5cm}

Therefore,

$f(P,Q,R) = m_3 + m_5 + m_6 + m_7$

\vspace{1cm}

\section*{Code Uploading Steps}

1. Create a PlatformIO project.  
2. Write the code in main.cpp inside src folder.  
3. Run the project using `pio run`.  
4. Upload using `pio run --target upload`.  
5. Connect Arduino UNO with OTG cable.  
6. Observe LED output and verify truth table.

\vspace{1cm}

\section*{Hardware Implementation}

The majority function is implemented using Arduino UNO.  
Inputs P, Q, R are connected to digital pins.  
Output F is connected to LED.

\vspace{0.5cm}

\section*{Required Components \& Pin Connections}

\begin{tabular}{|c|c|}
\hline
Component & Arduino Pin \\
\hline
Input P & Digital 2 \\
Input Q & Digital 3 \\
Input R & Digital 4 \\
Output F (LED) & Digital 8 \\
GND & GND \\
VCC & 5V \\
\hline
\end{tabular}

\vspace{1cm}

\section*{Logic Description}

• Initialize P, Q, R as digital inputs.  
• Compute:

$F = (P \cdot Q) + (Q \cdot R) + (P \cdot R)$  

• Display result on LED.  
• Change inputs as per truth table and verify output.

\vspace{1cm}

\section*{Experimental Truth Table}

\[
\begin{array}{c c c c}
P & Q & R & F(LED) \\
0 & 0 & 0 & 0 \\
0 & 0 & 1 & 0 \\
0 & 1 & 0 & 0 \\
0 & 1 & 1 & 1 \\
1 & 0 & 0 & 0 \\
1 & 0 & 1 & 1 \\
1 & 1 & 0 & 1 \\
1 & 1 & 1 & 1 \\
\end{array}
\]

\vspace{1cm}

\section*{Conclusion}

• The function outputs 1 when at least two inputs are 1.  
• Experimental results match theoretical truth table.  
• Hardware implementation confirms majority logic.

\end{document}
