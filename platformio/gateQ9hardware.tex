\documentclass[12pt,a4paper]{article}
\usepackage{graphicx}
\usepackage{geometry}
\usepackage{amsmath}
\usepackage{enumitem}
\geometry{margin=1in}

\begin{document}

\begin{center}
\includegraphics[width=5cm]{/storage/emulated/0/DCIM/MyAlbums/documents/IMG_20260202_154637.jpg}
\end{center}

\vspace{0.5cm}

\begin{center}
\textbf{Name:} Keshava Reddy V \\
\textbf{ID:} COMETFWC054
\end{center}
\vspace{0.8cm}

\begin{center}
\Large \textbf{GATE CS 2010 -- Question 9}
\end{center}

\vspace{0.8cm}

\section*{Question}

The Boolean expression for the output $f$ of the multiplexer shown below is:

\vspace{0.5cm}

\begin{center}
\includegraphics[width=8cm]{images/IMG_20260220_161244.jpg}
\end{center}

\vspace{0.5cm}

\textbf{Options:}

\begin{itemize}
\item[(a)] $P \oplus Q \oplus R$
\item[(b)] $\overline{P \oplus Q \oplus R}$
\item[(c)] $P + Q + R$
\item[(d)] $\overline{P + Q + R}$
\end{itemize}

\vspace{0.8cm}

\section*{Question Analysis}

The given circuit is a 4:1 Multiplexer.

\textbf{Select lines:} $P$ and $Q$

\textbf{Inputs:}
\begin{itemize}
\item $I_0 = R$
\item $I_1 = \overline{R}$
\item $I_2 = \overline{R}$
\item $I_3 = R$
\end{itemize}

\textbf{General 4:1 MUX equation:}

$f = \overline{P}\overline{Q}I_0 + \overline{P}QI_1 + P\overline{Q}I_2 + PQI_3$

\textbf{Substituting values:}

$f = \overline{P}\overline{Q}R + \overline{P}Q\overline{R} + P\overline{Q}\overline{R} + PQR$

\textbf{Grouping terms:}

$f = R(\overline{P}\overline{Q} + PQ) + \overline{R}(\overline{P}Q + P\overline{Q})$

Using identities:

$\overline{P}\overline{Q} + PQ = P \odot Q$

$\overline{P}Q + P\overline{Q} = P \oplus Q$

Therefore,

$f = P \oplus Q \oplus R$

\vspace{0.8cm}

\section*{Truth Table}

\begin{center}
\begin{tabular}{|c|c|c|c|}
\hline
P & Q & R & f \\
\hline
0 & 0 & 0 & 0 \\
0 & 0 & 1 & 1 \\
0 & 1 & 0 & 1 \\
0 & 1 & 1 & 0 \\
1 & 0 & 0 & 1 \\
1 & 0 & 1 & 0 \\
1 & 1 & 0 & 0 \\
1 & 1 & 1 & 1 \\
\hline
\end{tabular}
\end{center}

\vspace{0.8cm}

\section*{Circuit Diagram}

\begin{center}
\includegraphics[width=10cm]{images/IMG_20260220_153429.jpg}
\end{center}
\vspace{0.8cm}

\section*{Required Components}

\begin{itemize}
\item Arduino UNO
\item Breadboard
\item 7-Segment Display (Common Anode)
\item IC 7447
\item Connecting wires
\item resistor
\item Switches
\end{itemize}

\vspace{0.8cm}

\section*{Pin Connections}

\textbf{Input Connections:}

\begin{itemize}
\item $P \rightarrow$ Digital Pin 2
\item $Q \rightarrow$ Digital Pin 3
\item $R \rightarrow$ Digital Pin 4
\end{itemize}

\textbf{7447 Connections:}

\begin{itemize}
\item A $\rightarrow$ Pin 8
\item B $\rightarrow$ Pin 9
\item C $\rightarrow$ Pin 10
\item D $\rightarrow$ Pin 11
\item Common Anode $\rightarrow$ +5V
\end{itemize}

\vspace{0.8cm}

\section*{Code Uploading Steps}

\begin{enumerate}
\item Create a PlatformIO project.
\item Write the code in \texttt{src/main.cpp}.
\item Run the command \texttt{pio run}.
\item Copy the generated .hex file.
\item Connect Arduino UNO using OTG cable.
\item Upload using “Upload Precompiled” option.
\item Observe the output on the 7-segment display.
\end{enumerate}

\vspace{0.8cm}

\section*{Conclusion}

\item The output obtained matches the truth table.

\item Hence the Boolean expression of the given multiplexer is:

\begin{itemize}
\item Correct option: (a)   $ f = P \oplus Q \oplus R$
\end{itemize}
\end{document}
