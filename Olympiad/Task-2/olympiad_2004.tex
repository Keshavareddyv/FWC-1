\documentclass[12pt]{article}

\usepackage[draft=false]{graphicx}
\usepackage{amsmath}
\usepackage{amssymb}

\setlength{\parindent}{0pt}

\begin{document}

\begin{center}
\includegraphics[width=4cm]{iiitb_logo.png}
\end{center}

\begin{center}
cKeshava Reddy V\\
ID: COMETFWC054\\
45rd IMO 2004
\end{center}

\textbf{Problem 1.}
Let $ABC$ be an acute-angled triangle with $AB \neq AC$.
The circle with diameter $BC$ intersects the sides $AB$ and $AC$ at $M$ and $N$
respectively.
Denote by $O$ the midpoint of the side $BC$.
The bisectors of the angles $BAC$ and $MON$ intersect at $R$.
Prove that the circumcircles of the triangles $BMR$ and $CNR$
have a common point lying on the side $BC$.

\textbf{Problem 2.}
Find all polynomials $f$ with real coefficients such that for all real numbers
$a,b,c$ such that $ab+bc+ca=0$ we have the relation  
$f(a-b)+f(b-c)+f(c-a)=2f(a+b+c)$.

\textbf{Problem 3.}
Define a ``hook'' to be a figure made up of six unit squares as shown below
in the picture, or any of the figures obtained by applying rotations and
reflections to this figure.

\begin{center}
\includegraphics[width=3cm]{IMG_20260129_164949.jpg}
\end{center}

Determine all $m \times n$ rectangles that can be covered without gaps and
without overlaps with hooks such that
\begin{itemize}
\item the rectangle is covered without gaps and without overlaps
\item no part of a hook covers area outside the rectangle.
\end{itemize}

\textbf{Problem 4.}
Let $n \ge 3$ be an integer.
Let $t_1,t_2,\ldots,t_n$ be positive real numbers such that
\[
n^2+1>(t_1+t_2+\cdots+t_n)
\left(\frac{1}{t_1}+\frac{1}{t_2}+\cdots+\frac{1}{t_n}\right).
\]
Show that $t_i,t_j,t_k$ are side lengths of a triangle for all
$i,j,k$ with $1 \le i<j<k \le n$.

\textbf{Problem 5.}
In a convex quadrilateral $ABCD$ the diagonal $BD$ does not bisect the angles
$ABC$ and $CDA$.
The point $P$ lies inside $ABCD$ and satisfies  
$PBC = DBA$ and $PDC = BDA$.
Prove that $ABCD$ is a cyclic quadrilateral if and only if $AP = CP$.

\textbf{Problem 6.}
We call a positive integer alternating if every two consecutive digits in its
decimal representation are of different parity.
Find all positive integers $n$ such that $n$ has a multiple which is alternating.

\begin{center}

\end{center}

\end{document}
