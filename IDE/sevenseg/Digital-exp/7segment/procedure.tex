\documentclass[12pt]{article}
\usepackage{graphicx}

\begin{document}

%------------------------------------------------
% Logo
%------------------------------------------------
\begin{center}
\includegraphics[width=0.4\textwidth]{iitb_comet_logo.jpg}
\end{center}

\vspace{0.5cm}

\begin{center}
{\Large \textbf{Digital Electronics Lab}}\\
\vspace{0.2cm}
{\large \textbf{Seven Segment Display using Arduino}}\\
\vspace{0.3cm}
\textbf{SRINIDHI A}\\
COMETFWC053
\end{center}

\vspace{0.5cm}

%------------------------------------------------
\section*{Aim}
To interface a seven segment display with Arduino and verify the output.

%------------------------------------------------
\section*{Components Required}
\begin{itemize}
\item Arduino Uno
\item Seven segment display
\item Breadboard
\item Resistors
\item Connecting wires
\end{itemize}

%------------------------------------------------
\section*{Theory}
A seven segment display is an electronic display device used to
represent decimal numbers. It contains seven LEDs labelled a to g.
By turning ON specific segments, digits from 0 to 9 can be formed.

In a common anode configuration, the common terminal is connected
to the supply voltage and a LOW signal is required to glow a segment.
HIGH signal keeps the segment OFF.

%------------------------------------------------
\section*{Circuit Connections}
The segment pins were connected to Arduino digital output pins
through current limiting resistors. The common terminal of the
display was connected to Vcc. Proper grounding was ensured.

%------------------------------------------------
\section*{Procedure}
\begin{enumerate}
\item Placed the seven segment display on the breadboard carefully.
\item Checked the pin configuration of the display.
\item Identified the common terminal.
\item Connected the common terminal to Vcc.
\item Connected each segment pin (a to g) to Arduino digital pins through resistors.
\item Verified that all wiring connections were tight.
\item Connected the Arduino board to the mobile using USB cable.
\item Opened the Arduino programming environment.
\item Uploaded the program to the Arduino board.
\item Powered the circuit.
\item Observed the segments glowing according to the program.
\item Compared the output with the expected digit.
\end{enumerate}

%------------------------------------------------
\section*{Hardware Setup}
\begin{center}
\includegraphics[width=0.7\textwidth]{hardware.jpg}
\end{center}

%------------------------------------------------
\section*{Result}
The required digit was displayed successfully on the seven segment display.

%------------------------------------------------
\section*{Learning Outcome}
Understood the working of a seven segment display and learned how
Arduino output signals can control individual LED segments to
generate decimal numbers.

\end{document}
