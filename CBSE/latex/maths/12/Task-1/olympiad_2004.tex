\documentclass[12pt]{article}

\usepackage{graphicx}
\usepackage{amsmath}
\usepackage{amssymb}

\setlength{\parindent}{0pt}

\begin{document}

\begin{center}
\includegraphics[width=5cm]{IMG-20260130-WA0000.jpg}

Keshava Reddy V\\
ID: COMETFWC054\\
45th International Mathematical Olympiad 2004
\end{center}

\textbf{Problem 1.}

Let $ABC$ be an acute-angled triangle with $AB$ not equal to $AC$.
The circle with diameter $BC$ intersects the sides $AB$ and $AC$ at $M$ and $N$
respectively.
Let $O$ be the midpoint of $BC$.
The bisectors of the angles $BAC$ and $MON$ intersect at $R$.
Prove that the circumcircles of triangles $BMR$ and $CNR$
have a common point on the side $BC$.

\textbf{Problem 2.}

Find all polynomials $f$ with real coefficients such that for all real numbers
$a, b, c$ satisfying $ab + bc + ca = 0$,

$f(a - b) + f(b - c) + f(c - a) = 2 f(a + b + c)$.

\textbf{Problem 3.}

Define a hook to be a figure made up of six unit squares as shown below,
or any figure obtained by rotations and reflections of this figure.

\begin{center}
\includegraphics[width=4cm]{IMG_20260129_164949.jpg}
\end{center}

Determine all rectangles of size $m$ by $n$ that can be covered using hooks such that:
\begin{itemize}
\item the rectangle is covered without gaps and without overlaps;
\item no part of a hook lies outside the rectangle.
\end{itemize}

\newpage

\textbf{Problem 4.}

Let $n >= 3$ be an integer.
Let $t_1, t_2, t_3, t_4, t_5$ up to $t_n$ be positive real numbers such that

$n^2 + 1 >
(t_1 + t_2 + t_3 + t_4 + t_5 + t_n)
(1/t_1 + 1/t_2 + 1/t_3 + 1/t_4 + 1/t_5 + 1/t_n)$.

Show that $t_i, t_j, t_k$ are side lengths of a triangle for all
$i, j, k$ with $1 <= i < j < k <= n$.

\textbf{Problem 5.}

In a convex quadrilateral $ABCD$, the diagonal $BD$ does not bisect the angles
$ABC$ and $CDA$.
A point $P$ lies inside $ABCD$ such that
$PBC = DBA$ and $PDC = BDA$.
Prove that $ABCD$ is cyclic if and only if $AP = CP$.

\textbf{Problem 6.}

A positive integer is called alternating if every two consecutive digits
in its decimal representation have different parity.
Find all positive integers $n$ such that $n$ has a multiple which is alternating.

\end{document}
