\documentclass[12pt,a4paper]{article}
\usepackage{graphicx}
\usepackage{geometry}
\usepackage{amsmath}
\geometry{margin=1in}

\begin{document}
\begin{center}
\includegraphics[width=5cm]{/storage/emulated/0/DCIM/MyAlbums/documents/IMG_20260202_154637.jpg}
\end{center}

\vspace{0.3cm}

\begin{center}
\textbf{Name:} Keshava Reddy V\\
\textbf{ID:} COMETFWC054
\end{center}
\vspace{0.5cm}


\begin{center}
\Large \textbf{GATE CS 2010}
\end{center}

\vspace{0.5cm}
\textbf{Question 9}

\vspace{0.3cm}

The Boolean expression for the output $f$ of the multiplexer shown below is:

\vspace{0.3cm}

\includegraphics[width=8cm]{IMG_20260220_122317.jpg}

\vspace{0.5cm}

\textbf{Options}

(a) $P \oplus Q \oplus R$  

(b) $\overline{P \oplus Q \oplus R}$  

(c) $P + Q + R$  

(d) $\overline{P + Q + R}$  

\vspace{0.8cm}

\textbf{Solution}

\vspace{0.3cm}

The given circuit is a 4:1 Multiplexer.

Select lines: $P$ and $Q$.

Inputs:

$I_0 = R$  

$I_1 = \overline{R}$  

$I_2 = \overline{R}$  

$I_3 = R$  

\vspace{0.4cm}

General equation of 4:1 MUX:

$f = \overline{P}\overline{Q}I_0 + \overline{P}QI_1 + P\overline{Q}I_2 + PQI_3$

\vspace{0.4cm}

Substituting values:

$f = \overline{P}\overline{Q}R + \overline{P}Q\overline{R} + P\overline{Q}\overline{R} + PQR$

\vspace{0.4cm}

Grouping terms:

$f = R(\overline{P}\overline{Q} + PQ) + \overline{R}(\overline{P}Q + P\overline{Q})$

\vspace{0.4cm}

Using identities:

$\overline{P}\overline{Q} + PQ = P \odot Q$

$\overline{P}Q + P\overline{Q} = P \oplus Q$

\vspace{0.4cm}

Therefore,

$f = P \oplus Q \oplus R$

\vspace{0.8cm}

\textbf{Final Answer}

$f = P \oplus Q \oplus R$

Correct option: (a)

\end{document}
