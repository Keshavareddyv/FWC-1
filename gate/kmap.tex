\documentclass[a4paper,12pt]{article}

\usepackage{graphicx}
\usepackage{amsmath}
\usepackage{array}
\usepackage{geometry}
\usepackage{enumitem}

\geometry{margin=1in}

\begin{document}

\begin{center}

\includegraphics[width=5cm]{/storage/emulated/0/DCIM/MyAlbums/documents/IMG_20260202_154637.jpg}

\vspace{0.5cm}

Name: Keshava Reddy V \\
ID: COMETFWC054

\vspace{0.5cm}

{\Large \textbf{GATE CS 2010}}

\end{center}

\vspace{1cm}

\section*{Question 6}

{ The minterm expansion of $f(P,Q,R) = PQ + QR + PR$ is}

\vspace{0.3cm}

\begin{enumerate}
\item $m_2 + m_4 + m_6 + m_7$
\item $m_0 + m_1 + m_3 + m_5$
\item $m_0 + m_1 + m_6 + m_7$
\item $m_2 + m_3 + m_4 + m_5$
\end{enumerate}


\section*{Question Analysis}

Given function: $f(P,Q,R) = PQ + QR + PR$


\vspace{0.4cm}
\begin{itemize}
\item This is a 3-input Majority Function.
\item The output becomes 1 when at least two inputs are 1.
\vspace{1cm}
\end{itemize}



\section*{Truth Table}

\begin{center}
\begin{tabular}{|c|c|c|c|}
\hline
P & Q & R & F \\
\hline
0 & 0 & 0 & 0 \\
0 & 0 & 1 & 0 \\
0 & 1 & 0 & 0 \\
0 & 1 & 1 & 1 \\
1 & 0 & 0 & 0 \\
1 & 0 & 1 & 1 \\
1 & 1 & 0 & 1 \\
1 & 1 & 1 & 1 \\
\hline
\end{tabular}
\end{center}

\vspace{0.6cm}

Minterms where $F = 1$:

$m_3 + m_5 + m_6 + m_7$

\vspace{0.4cm}

Therefore,

$f(P,Q,R) = m_3 + m_5 + m_6 + m_7$

\vspace{1cm}

\section*{Hardware Implementation}
\begin{itemize}
\item The above Boolean function is implemented using Arduino UNO and a 7447 BCD to Seven Segment Decoder IC.
\item The logical output of the function is given as input to the 7447 IC, which drives a Common Anode Seven Segment Display.
\item The 7447 converts the BCD input into corresponding segment signals, and 220$\Omega$ current limiting resistors are used to protect the display segments.
\item The hardware setup verifies the theoretical Majority Function experimentally.
\end{itemize}
\vspace{1cm}


\section*{Required Components}

\begin{enumerate}
\item Arduino UNO
\item Breadboard
\item Seven Segment Display (Common Cathode)
\item 7447
\item Jumper Wires
\item USB Cable
\end{enumerate}

\vspace{1cm}

\section*{Pin Connections}

\begin{center}
\begin{tabular}{|c|c|}
\hline
Component & Arduino Pin \\
\hline
Input P & Digital 2 \\
Input Q & Digital 3 \\
Input R & Digital 4 \\
Segment a & Digital 5 \\
Segment b & Digital 6 \\
Segment c & Digital 7 \\
Segment d & Digital 8 \\
Segment e & Digital 9 \\
Segment f & Digital 10 \\
Segment g & Digital 11 \\
Common Cathode & GND \\
\hline
\end{tabular}
\end{center}

\vspace{1cm}

\section*{Circuit Diagram}

\begin{center}

\includegraphics[width=14cm]{/storage/emulated/0/DCIM/Camera/IMG_20260219_111637.jpg}

\end{center}

\vspace{1cm}

\section*{Conclusion}
\begin{itemize}
\item The function outputs 1 when at least two inputs are 1.
\item The hardware implementation verifies the theoretical Majority Function and matches the minterm expansion:
\item $m_3 + m_5 + m_6 + m_7$
\end{itemize}
\end{document}
